\documentclass[aspectratio=169]{beamer}

% Pacchetti necessari
\usepackage[utf8]{inputenc}
\usepackage[T1]{fontenc}
\usepackage[italian]{babel}
\usepackage{tikz}
\usepackage{graphicx}
\usepackage{xcolor}
\usetikzlibrary{positioning, calc, backgrounds, shapes.geometric}

% --- DEFINIZIONE COLORI (Light Mode - Stile Report) ---
\definecolor{ReportWhite}{HTML}{FFFFFF}    % Sfondo Bianco
\definecolor{ReportDarkBlue}{HTML}{003366} % Blu scuro
\definecolor{ReportBlack}{HTML}{212121}    % Testo scuro
\definecolor{ReportGold}{HTML}{D4AF37}     % Oro (accenti)

% --- SETUP BEAMER ---
\setbeamercolor{background canvas}{bg=ReportWhite}
\setbeamercolor{normal text}{fg=ReportBlack}
\setbeamercolor{frametitle}{fg=ReportDarkBlue, bg=ReportWhite}
\setbeamercolor{title}{fg=ReportDarkBlue}
\setbeamercolor{subtitle}{fg=ReportBlack}
\setbeamercolor{itemize item}{fg=ReportDarkBlue}

% Font
\usefonttheme{structurebold}

% --- METADATI ---
\title{\textbf{GomorraSQL Compiler}}
\subtitle{Compilatore SQL con Dialetto Napoletano $\to$ LLVM IR}
\author{Angelo Alberico}
\date{11 Gennaio 2026}
\newcommand{\matricola}{Matr. NF22500104} 
\newcommand{\corsodistudi}{Corso di ILP}

% --- HEADER: NUMERO DI PAGINA IN ALTO A DESTRA ---
\setbeamertemplate{headline}{
  \ifnum\insertframenumber>1 % Nascondi sulla prima pagina
    \vspace{0.2cm} % Spazio dal bordo superiore
    \hfill % Spinge tutto a destra
    \color{gray}\footnotesize\insertframenumber{} / \inserttotalframenumber\hspace{0.5cm} % Numero e margine destro
    \vspace{0.1cm}
  \fi
}

% --- FOOTER: SOLO INFO, NIENTE NUMERO PAGINA ---
\setbeamertemplate{footline}{
  \ifnum\insertframenumber>1
  \leavevmode%
  \hbox{%
  % Box Sinistro: Autore e Matricola
  \begin{beamercolorbox}[wd=.5\paperwidth,ht=2.5ex,dp=1.125ex,leftskip=.3cm]{author in head/foot}%
    \color{gray}\insertauthor \ - \matricola
  \end{beamercolorbox}%
  % Box Destro: Corso di Studi (allineato a destra)
  \begin{beamercolorbox}[wd=.5\paperwidth,ht=2.5ex,dp=1.125ex,rightskip=.3cm, right]{title in head/foot}%
    \color{gray}\corsodistudi
  \end{beamercolorbox}}%
  \vskip0pt%
  \fi
}

% Rimuove i simboli di navigazione standard
\setbeamertemplate{navigation symbols}{}

% --- COMANDO SLIDE ---
\newcommand{\GomorraSlide}[4]{
\begin{frame}{#1}
    \begin{columns}[T]
        % Colonna Sinistra (Contenuto)
        \begin{column}{0.75\textwidth}
            #4
        \end{column}
        
        % Colonna Destra (Spazio vuoto)
        \begin{column}{0.25\textwidth}
        \end{column}
    \end{columns}

    % Overlay TikZ per Immagine e Testo
    \begin{tikzpicture}[remember picture, overlay]
        % Immagine
        \node[anchor=south east, inner sep=0pt, xshift=0.5cm, yshift=-0.5cm] (char) at (current page.south east) {
            \reflectbox{\includegraphics[height=0.55\paperheight, keepaspectratio]{#2}}
        };
        
        % Testo: Piccolo, corsivo, tra virgolette, sopra la testa
        \ifx&#3&%
        \else
            \node[
                anchor=south,             
                text width=4cm,           
                align=center,             
                font=\footnotesize\itshape, 
                text=ReportBlack,         
                yshift=0.1cm              
            ] at (char.north) {           
                ``#3''
            };
        \fi
    \end{tikzpicture}
\end{frame}
}

% --- 2. NUOVO COMANDO: SLIDE SEMPLICE (SimpleSlide) ---
% Nessuna immagine, layout a tutta larghezza
% #1: Titolo
% #2: Contenuto
\newcommand{\SimpleSlide}[2]{
\begin{frame}{#1}
    #2
\end{frame}
}

% --- INIZIO DOCUMENTO ---
\begin{document}

% --- 1. FRONT PAGE ---
{
\usebackgroundtemplate{
    \begin{tikzpicture}[remember picture, overlay]
        \fill[ReportWhite] (current page.south west) rectangle (current page.north east);
        \fill[ReportDarkBlue] (current page.north west) rectangle ([xshift=0.5cm]current page.south west);
        \node[anchor=north east, opacity=0.03, text=black, align=right, font=\ttfamily\tiny] at (current page.north east) {
            1010101010 0010101011 1101010100\\
            0101010100 1110101010 0001010111\\
            SELECT * FROM users WHERE id = 1
        };
    \end{tikzpicture}
}
\begin{frame}[plain]
    \centering
    \vspace{0.5cm}
    
    {\Huge \textbf{\textcolor{ReportDarkBlue}{GomorraSQL}}}\\[0.2cm]
    \textcolor{ReportGold}{\rule{5cm}{1pt}} \\[0.2cm]
    {\Large \textcolor{ReportBlack}{Neapolitan Dialect Compiler}}\\[0.1cm]
    {\small \textcolor{gray}{\textit{Quando 'o SQL parla napoletano}}}
    
    \vspace{1.0cm}
    \includegraphics[height=2.5cm]{gomorra_gun.png} 
    \vspace{1.0cm}
    
    \begin{columns}
        \column{0.5\textwidth}
        \raggedleft
        \textbf{\insertauthor} \\
        \footnotesize \matricola
        
        \column{0.5\textwidth}
        \raggedright
        \footnotesize 
        \textbf{Implementation Report} \\
        \insertdate
    \end{columns}
\end{frame}
}

% --- 2. INTRODUZIONE ---
\GomorraSlide{Introduzione: Progetto Originale}
{gomorra_teach.png}
{Chest' è 'o progetto nuost'.}
{
    \textbf{GomorraSQL}: DSL dichiarativo ispirato a SQL con dialetto napoletano.
    
    \vspace{0.5cm}
    
    \textbf{Caratteristiche principali:}
    \begin{itemize}
        \item \textbf{Traduttore runtime:} GomorraSQL $\to$ SQL standard
        \item Esecuzione su database esistenti (MySQL, PostgreSQL, SQLite)
        \item Implementato in Java con JDBC
        \item Nessuna analisi semantica o ottimizzazione
    \end{itemize}
    
    \vspace{0.5cm}
    
    \textbf{Ispirazione:} 
    
    \vspace{0.2cm}
    \includegraphics[height=0.4cm]{github-logo.png} \hspace{0.2cm} \textcolor{ReportDarkBlue}{\texttt{github.com/aurasphere/gomorra-sql}}
}

% --- 3. ISPIRAZIONE: GOMORRA-SQL ORIGINALE ---
\GomorraSlide{Introduzione: Il nostro progetto}
{gomorra_face.png}
{Chest' è 'o progetto nuost'.}
{    
    \textbf{Architettura:}
    \begin{itemize}
        \item Parsing e validazione semantica di query SELECT
        \item Esecuzione diretta su \textbf{file CSV}, non database        
        \item Generazione codice LLVM IR
    \end{itemize}
    
    \vspace{0.4cm}
    
    \textbf{Esempio:}
    \begin{center}
        \texttt{\small ripigliammo * mmiez 'a "guaglioni.csv"} \\
        $\downarrow$ \\
        \texttt{\small SELECT * FROM guaglioni}
    \end{center}
}

% --- 3.5 GENERAZIONE LLVM IR ---
\GomorraSlide{Generazione LLVM IR}
{gomorra_gun.png}
{Facimm 'o core d' 'a query.}
{
    \textbf{Strategia di Compilazione:}
    
    \vspace{0.4cm}
    
    La generazione di codice LLVM IR si concentra \textbf{esclusivamente} sulla clausola \texttt{WHERE}.
    
    \vspace{0.4cm}
    
    \textbf{Cosa viene compilato:}
    \begin{itemize}
        \item Condizioni logiche (\texttt{e}, \texttt{o})
        \item Operatori di confronto (\texttt{>}, \texttt{<}, \texttt{=}, \texttt{<>})
        \item Controlli NULL (\texttt{è nisciun}, \texttt{nun è nisciun})
    \end{itemize}
    
    \vspace{0.4cm}
    
    \textbf{Resto della query:} Parsing, validazione semantica, caricamento CSV in Python.
}

% --- 6. ANALISI SINTATTICA ---
\SimpleSlide{Strumenti e Tecnologie}
{
    \begin{columns}[c]
        % Colonna 1: Lark Parser
        \begin{column}{0.25\textwidth}
            \centering
            {\Huge \textcolor{ReportDarkBlue}{\textbf{\{~\}}}} \\[0.5cm]
            \textbf{\textcolor{ReportDarkBlue}{Parser}} \\[0.3cm]
            \small Lark (Python) \\
            Grammatica EBNF \\
            Lexer + Parser
        \end{column}
        
        % Colonna 2: llvmlite
        \begin{column}{0.25\textwidth}
            \centering
            {\Huge \textcolor{ReportDarkBlue}{\textbf{$\gg$}}} \\[0.5cm]
            \textbf{\textcolor{ReportDarkBlue}{Code Generator}} \\[0.3cm]
            \small llvmlite \\
            LLVM IR \\
            Ottimizzazione
        \end{column}
        
        % Colonna 3: Python
        \begin{column}{0.25\textwidth}
            \centering
            {\Huge \textcolor{ReportDarkBlue}{\textbf{Py}}} \\[0.5cm]
            \textbf{\textcolor{ReportDarkBlue}{Linguaggio}} \\[0.3cm]
            \small Python 3.11 \\
            Implementazione \\
            Testing (pytest)
        \end{column}
        
    \end{columns}
}

% --- 4. PIPELINE DI COMPILAZIONE ---
\SimpleSlide{Pipeline di Compilazione}
{
    \textbf{Fasi di Compilazione:} dalla query in formato testuale fino alla generazione di codice LLVM IR.
    
    \vspace{0.3cm}
    
    La pipeline prevede una fase di \textbf{analisi semantica} che valida la correttezza della query prima di procedere alla generazione del codice.
    
    
    \begin{center}
    \begin{tikzpicture}[
        node distance=1.5cm and 1.2cm,
        every node/.style={font=\tiny},
        box/.style={rectangle, draw=ReportDarkBlue, fill=ReportDarkBlue!10, thick, minimum width=1.6cm, minimum height=0.8cm, rounded corners},
        decision/.style={diamond, draw=ReportGold, fill=ReportGold!20, thick, text width=1.3cm, align=center, inner sep=1pt, minimum height=1cm},
        arrow/.style={->, >=stealth, thick, ReportDarkBlue}
    ]
        % Nodi principali - disposizione orizzontale
        \node[box, align=center] (query) {Query\\(String)};
        \node[box, right=of query, align=center] (parser) {Parser\\(Lark\\LALR)};
        \node[box, right=of parser, align=center] (transformer) {Transformer\\(AST)};
        \node[decision, right=of transformer] (semantic) {Semantic Analyzer};
        \node[box, right=of semantic, align=center] (llvm) {LLVM\\CodeGen};
        
        % Errore
        \node[box, below=0.8cm of semantic, fill=red!20, draw=red, minimum width=1.5cm] (error) {Error};
        
        % Frecce
        \draw[arrow] (query) -- (parser);
        \draw[arrow] (parser) -- (transformer);
        \draw[arrow] (transformer) -- (semantic);
        \draw[arrow] (semantic) -- node[above, font=\tiny] {OK} (llvm);
        \draw[arrow] (semantic) -- (error);
    \end{tikzpicture}
    \end{center}
}




% --- 6. ANALISI LESSICALE ---
\GomorraSlide{Analisi Lessicale}
{gomorra_gun.png}
{'O Lexer nun scherza.}
{
    \textbf{Token Recognition:} Keywords napoletane $\to$ Token
    
    \vspace{0.3cm}
    
    \begin{small}
    \begin{tabular}{ll}
        \texttt{ripigliammo} & $\to$ SELECT\_KW \\
        \texttt{mmiez 'a} & $\to$ FROM\_KW \\
        \texttt{arò} & $\to$ WHERE\_KW \\
        \texttt{e} / \texttt{o} & $\to$ AND\_KW / OR\_KW \\
        \texttt{è nisciun} & $\to$ IS\_KW NULL\_KW \\
        \texttt{>=, <=, <>, =} & $\to$ COMP\_OP
    \end{tabular}
    \end{small}
    
}


% --- 7. ANALISI SINTATTICA: GRAMMATICA ---
\begin{frame}[fragile]
\frametitle{Analisi Sintattica: Grammatica EBNF}
\begin{columns}[T]
\begin{column}{0.75\textwidth}
    \textbf{Parser LALR con Lark}
    
    \vspace{0.2cm}
    
    \begin{small}
    \begin{verbatim}
select_stmt: SELECT_KW projection 
             from_clause [where_clause]

projection: ALL_COLS | column_list

from_clause: FROM_KW table_ref join_clause*

where_clause: WHERE_KW condition

condition: logic_term (OR_KW logic_term)*
logic_term: logic_factor (AND_KW logic_factor)*
logic_factor: comparison | null_check 
            | "(" condition ")"
    \end{verbatim}
    \end{small}
\end{column}
\begin{column}{0.25\textwidth}
\end{column}
\end{columns}
\begin{tikzpicture}[remember picture, overlay]
\node[anchor=south east, inner sep=0pt, xshift=0.5cm, yshift=-0.5cm] (char) at (current page.south east) {
\reflectbox{\includegraphics[height=0.55\paperheight, keepaspectratio]{gomorra_teach.png}}
};
\node[anchor=south, text width=4cm, align=center, font=\footnotesize\itshape, text=ReportBlack, yshift=0.1cm] at (char.north) {'A grammatica è 'a legge.};
\end{tikzpicture}
\end{frame}


% % --- 8. AST NODE STRUCTURE ---
% \begin{frame}[fragile]{Struttura AST Nodes}
%     \begin{columns}[t]
%         \begin{column}{0.5\textwidth}
%             \textbf{Nodo Radice:}
%             \begin{small}
%             \begin{verbatim}
% @dataclass
% class SelectQuery:
%     columns: List[str] | "*"
%     tables: List[str]
%     where: Condition | None
%             \end{verbatim}
%             \end{small}
            
%             \vspace{0.3cm}
            
%             \textbf{Condizioni:}
%             \begin{small}
%             \begin{verbatim}
% @dataclass
% class Comparison(Condition):
%     left: str
%     operator: str
%     right: str | int | float
%             \end{verbatim}
%             \end{small}
%         \end{column}
        
%         \begin{column}{0.5\textwidth}
%             \textbf{Operatori Logici:}
%             \begin{small}
%             \begin{verbatim}
% @dataclass
% class LogicOp(Condition):
%     operator: 'AND' | 'OR'
%     conditions: List[Condition]

% @dataclass
% class NullCheck(Condition):
%     column: str
%     is_null: bool
%             \end{verbatim}
%             \end{small}
%         \end{column}
%     \end{columns}
% \end{frame}

% --- 9. AST TREE VISUALIZATION - PARTE 1: SELECTQUERY ---
\begin{frame}
\frametitle{Visualizzazione AST (1/2): Struttura SelectQuery}
    \textbf{Query:} \texttt{RIPIGLIAMMO nome MMIEZ 'A "guaglioni.csv" arò (eta $>$ 18 e zona = "Scampia") o nome = "Ciro"}
    
    \vspace{0.5cm}
    
    \begin{center}
    \begin{tikzpicture}[
        level distance=1.5cm,
        level 1/.style={sibling distance=5cm},
        every node/.style={draw, rectangle, rounded corners, font=\small}
    ]
        \node[fill=ReportDarkBlue!20] {\textbf{SelectQuery}}
            child {node[fill=ReportGold!20] {columns: [nome]}}
            child {node[fill=ReportGold!20] {tables: [guaglioni.csv]}}
            child {node[fill=ReportGold!20] {\textbf{WHERE Condition}}};
    \end{tikzpicture}
    \end{center}
    
    \vspace{0.5cm}
    
    \textbf{Componenti principali:}
    \begin{itemize}
        \item \textbf{Projection:} Lista di colonne da selezionare
        \item \textbf{From:} Tabelle sorgente (CSV files)
        \item \textbf{Where:} Albero condizioni logiche
    \end{itemize}
\end{frame}

% --- 9.5. AST TREE VISUALIZATION - PARTE 2: LOGICOP ---
\begin{frame}
\frametitle{Visualizzazione AST (2/2): Albero Condizioni WHERE}
    \textbf{Condizione:} \texttt{(eta $>$ 18 e zona = "Scampia") o nome = "Ciro"}
    
    \vspace{0.5cm}
    
    \begin{center}
    \begin{tikzpicture}[
        level distance=1.5cm,
        level 1/.style={sibling distance=6cm},
        level 2/.style={sibling distance=4cm},
        every node/.style={draw, rectangle, rounded corners, font=\small}
    ]
        \node[fill=ReportGold!20] {\textbf{LogicOp (OR)}}
            child {node[fill=green!20] {\textbf{LogicOp (AND)}}
                child {node[fill=blue!10] {Cms: eta > 18}}
                child {node[fill=blue!10] {Cms: zona = Scampia}}
            }
            child {node[fill=green!20] {Comparison: nome = Ciro}};
    \end{tikzpicture}
    \end{center}
    
    
    \textbf{Struttura gerarchica:}
    \begin{itemize}
        \item \textbf{OR esterno:} Valutazione short-circuit
        \item \textbf{AND interno:} Entrambe condizioni devono essere vere
        \item \textbf{Confronti:} Foglie dell'albero
    \end{itemize}
\end{frame}

% --- 10. ANALISI SEMANTICA: SCOPE ---
\begin{frame}[fragile]
\frametitle{Analisi Semantica: Symbol Table}
\begin{columns}[T]
\begin{column}{0.75\textwidth}
    \textbf{Validazione Colonne tramite CSV Header}
    
    \vspace{0.3cm}
    
    \textbf{Step 1:} Carica schema tabelle
    \begin{small}
    \begin{verbatim}
guaglioni.csv header → {nome, zona, eta}
    \end{verbatim}
    \end{small}
    
    \vspace{0.2cm}
    
    \textbf{Step 2:} Verifica esistenza colonne
    \begin{small}
    \begin{itemize}
        \item SELECT: \texttt{nome} $\in$ \{nome, zona, eta\} $\checkmark$
        \item WHERE: \texttt{eta} $\in$ \{nome, zona, eta\} $\checkmark$
        \item WHERE: \texttt{zona} $\in$ \{nome, zona, eta\} $\checkmark$
    \end{itemize}
    \end{small}
    
    \vspace{0.2cm}
    
    \textbf{Errore Semantico:}
    \begin{small}
    \texttt{RIPIGLIAMMO cognome MMIEZ 'A "guaglioni.csv" ...} $\to$ \textcolor{red}{SemanticError: Colonna 'cognome' non trovata}
    \end{small}
\end{column}
\begin{column}{0.25\textwidth}
\end{column}
\end{columns}
\begin{tikzpicture}[remember picture, overlay]
\node[anchor=south east, inner sep=0pt, xshift=0.5cm, yshift=-0.5cm] (char) at (current page.south east) {
\reflectbox{\includegraphics[height=0.55\paperheight, keepaspectratio]{gomorra_gun.png}}
};
\node[anchor=south, text width=4cm, align=center, font=\footnotesize\itshape, text=ReportBlack, yshift=0.1cm] at (char.north) {'E colonne se devono truvà.};
\end{tikzpicture}
\end{frame}

% --- 11. ANALISI SEMANTICA: JOIN ---
% \GomorraSlide{Analisi Semantica: JOIN Operations}
% {gomorra_face.png}
% {'E tabelle se devono azzeccà.}
% {
%     \textbf{Sintassi JOIN in GomorraSQL:}
    
%     \vspace{0.3cm}
    
%     \texttt{RIPIGLIAMMO ... MMIEZ 'A tab1.csv \textbf{pesc e pesc} tab2.csv}
    
%     \vspace{0.5cm}
    
%     \textbf{Implementazione:}
%     \begin{itemize}
%         \item \textbf{pesc e pesc:} INNER JOIN (prodotto cartesiano)
%         \item Join implicita: tutte le combinazioni di righe
%         \item Filtro opzionale con clausola \texttt{arò}
%     \end{itemize}
    
%     \vspace{0.5cm}
    
%     \textbf{Esempio:}
%     \begin{small}
%     \texttt{RIPIGLIAMMO nome, ruolo MMIEZ 'A guaglioni.csv}\\
%     \texttt{pesc e pesc ruoli.csv arò id = id\_2}
%     \end{small}
    
    
% }

% --- 12. ANALISI SEMANTICA: JOIN DISAMBIGUATION ---
% \begin{frame}[fragile]
% \frametitle{Analisi Semantica: Disambiguazione JOIN}
% \begin{columns}[T]
% \begin{column}{0.75\textwidth}
%     \textbf{Problema:} Due tabelle con colonna "nome"
    
    
%     \begin{small}
%     \begin{verbatim}
% guaglioni.csv: nome, zona, eta
% ruoli.csv: id, nome, ruolo
%     \end{verbatim}
%     \end{small}
    
    
%     \textbf{Soluzione Automatica:} Suffisso \_2
%     \begin{small}
%     \begin{verbatim}
% JOIN result: nome, zona, eta, id, nome_2, ruolo
%                                      ^
%                               suffisso aggiunto
%     \end{verbatim}
%     \end{small}
    
%     \vspace{0.3cm}
    
%     \textbf{Query Valida:}
%     \begin{small}
%     \texttt{RIPIGLIAMMO nome, nome\_2, ruolo MMIEZ 'A ... pesc e pesc ... arò nome = nome\_2}
%     \end{small}
% \end{column}
% \begin{column}{0.25\textwidth}
% \end{column}
% \end{columns}
% \begin{tikzpicture}[remember picture, overlay]
% \node[anchor=south east, inner sep=0pt, xshift=0.5cm, yshift=-0.5cm] (char) at (current page.south east) {
% \reflectbox{\includegraphics[height=0.55\paperheight, keepaspectratio]{gomorra_teach.png}}
% };
% \node[anchor=south, text width=4cm, align=center, font=\footnotesize\itshape, text=ReportBlack, yshift=0.1cm] at (char.north) {'E colonne duplicate se devono sistemà.};
% \end{tikzpicture}
% \end{frame}

% --- 13. VISITOR PATTERN ---
\GomorraSlide{Generazione LLVM: Visitor Pattern}
{gomorra_teach.png}
{'O Visitor cammina pe' ll'AST.}
{
    \textbf{Pattern Utilizzato:} Visitor Pattern per attraversare l'AST
    
    \vspace{0.3cm}
    
    \textbf{Implementazione:}
    \begin{itemize}
        \item \texttt{LLVMCodeGenerator} estende \texttt{ASTVisitor}
        \item Ogni nodo AST ha un metodo \texttt{visit\_*} dedicato
        \item Il metodo \texttt{visit()} fa dispatch automatico
    \end{itemize}
    
    \vspace{0.3cm}
    
    \textbf{Metodi Visitor:}
    \begin{small}
    \begin{itemize}
        \item \texttt{visit\_comparison()} $\to$ Genera codice per confronti
        \item \texttt{visit\_logic\_op()} $\to$ Genera AND/OR
        \item \texttt{visit\_null\_check()} $\to$ Genera controlli NULL
    \end{itemize}
    \end{small}
    
    \vspace{0.3cm}
    
    \textbf{Vantaggi:} Separazione tra struttura AST e generazione codice
}

% --- 14. LLVM: SCOPING E BLOCCHI ---
\GomorraSlide{Generazione LLVM: Struttura IR}
{gomorra_gun.png}
{'A funzione è semplice.}
{
    \textbf{Struttura del Codice LLVM IR:}
    
    \vspace{0.3cm}
    
    Il compilatore genera una \textbf{singola funzione LLVM} per valutare la clausola WHERE.
    
    \vspace{0.3cm}
    
    \textbf{Architettura:}
    \begin{itemize}
        \item \textbf{Un solo blocco "entry":} Tutta la logica WHERE in un blocco
        \item \textbf{IRBuilder:} Costruisce istruzioni sequenzialmente
        \item \textbf{Nessun branching:} Le condizioni AND/OR usano operatori bit-a-bit
        \item Termina con \texttt{ret i1} (ritorna true/false)
    \end{itemize}
    
}

% --- 15. TYPE INFERENCE: REGOLE ---
\GomorraSlide{Type Inference: Regole di Inferenza}
{gomorra_teach.png}
{'E tipi se scoprono da soli.}
{
    \textbf{Analisi Automatica CSV $\to$ Tipo LLVM:}
    
    
    Il compilatore analizza i valori del CSV per determinare il tipo di ogni colonna.
    
    
    \begin{center}
    \begin{tabular}{|l|l|l|}
        \hline
        \textbf{Condizione} & \textbf{Tipo Inferito} & \textbf{Esempio} \\
        \hline
        Valore vuoto & \texttt{NULL} & \texttt{""} \\
        \hline
        Contiene \texttt{.} e convertibile & \texttt{float} & \texttt{"19.99"} \\
        \hline
        Convertibile a intero & \texttt{int} & \texttt{"35"} \\
        \hline
        Altrimenti & \texttt{str} & \texttt{"Ciro"} \\
        \hline
    \end{tabular}
    \end{center}
    
    
    \textbf{Strategia:}
    \begin{itemize}
        \item Campiona prime 100 righe del CSV
        \item \texttt{float} prevale su \texttt{int} se presente
        \item Tipo predominante determina il tipo della colonna
    \end{itemize}
}

% --- 16. ESEMPIO LLVM IR: OPERATORI LOGICI ---
\begin{frame}[fragile]
\frametitle{Esempio LLVM IR: Operatori Logici AND/OR}
    \textbf{Query:} \texttt{arò (eta > 18 e zona = "Scampia") o nome = "Ciro"}
    
    
    \textbf{LLVM IR Generato:}
    
    
    \begin{small}
    \begin{verbatim}
define i1 @evaluate_row(i32 %".1") {entry:
  ; Primo confronto: eta > 18
  ; icmp_signed('>') → icmp sgt (signed greater than)
  %".3" = icmp sgt i32 35, 18
  
  ; AND logico con secondo confronto (zona = "Scampia")
  ; builder.and_() → and i1 (bitwise AND su booleani)
  %".4" = and i1 %".3", 1
  
  ; OR logico con terzo confronto (nome = "Ciro")
  ; builder.or_() → or i1 (bitwise OR su booleani)
  %".5" = or i1 %".4", 1
  ret i1 %".5"}
    \end{verbatim}
    \end{small}
        
\end{frame}

% --- 18. ESEMPIO LLVM IR: NULL CHECK ---
% \begin{frame}[fragile]
% \frametitle{Esempio LLVM IR: Controllo NULL}
%     \textbf{Query:} \texttt{RIPIGLIAMMO * MMIEZ 'A "guaglioni\_null.csv" arò zona è nisciun}
    
%     \vspace{0.3cm}
    
%     \textbf{LLVM IR Generato:}
    
%     \vspace{0.2cm}
    
%     \begin{small}
%     \begin{verbatim}
% define i1 @evaluate_row(i32 %".1") {
% entry:
%   ; Controllo NULL semplificato
%   ; In implementazione completa:
%   ; %ptr = load i8*, i8** %zona_ptr
%   ; %is_null = icmp eq i8* %ptr, null
  
%   ; Demo: ritorna sempre false (0)
%   ; visit_null_check() genera costante
%   ret i1 0
% }
%     \end{verbatim}
%     \end{small}
    
%     \vspace{0.3cm}
    
%     \textbf{Implementazione Completa:}
%     \begin{itemize}
%         \item Caricamento puntatore stringa: \texttt{load i8*}
%         \item Confronto con NULL: \texttt{icmp eq i8* \%ptr, null}
%         \item Tipo puntatore: \texttt{i8*} (char*)
%     \end{itemize}
% \end{frame}

% --- 19. TESTING E CODE COVERAGE ---
\GomorraSlide{Testing e Code Coverage}
{gomorra_teach.png}
{'E test nun mentono.}
{
    \textbf{19 test} che coprono tutti i casi d'uso principali:
    \begin{itemize}
        \item SELECT semplice e complesso
        \item WHERE con operatori logici (AND/OR)
        \item JOIN con disambiguazione
        \item Controlli NULL
        \item Errori semantici e sintattici
        \item Type inference e generators
    \end{itemize}
    
    \vspace{0.3cm}
    
    % \textbf{Code Coverage: 89.39\%}
    
    
    \textbf{Risultato:} Tutti i 19 test passati $\checkmark$
}

% --- 20. CONCLUSIONI ---
\begin{frame}[plain]
    \centering
    \vspace{1.5cm}
    
    {\Huge \textcolor{ReportDarkBlue}{\textbf{Grazie per l'attenzione}}}
    
    \vspace{1.5cm}
    
    \begin{columns}
        \column{0.5\textwidth}
        \centering
        \textbf{\Large Angelo Alberico}\\[0.3cm]
        \textcolor{gray}{Matr. NF22500104}\\[0.5cm]
        
        \includegraphics[height=0.4cm]{github-logo.png} \\[0.2cm]
        \textcolor{ReportDarkBlue}{\texttt{github.com/Nakura125}}
        
        \column{0.5\textwidth}
        \centering
        \includegraphics[height=4cm]{gomorra_gun.png}
    \end{columns}
    
    \vspace{0.5cm}
    
    \textcolor{gray}{\textit{GomorraSQL - Quando 'o SQL parla napoletano}}
\end{frame}

\end{document}